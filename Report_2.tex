\documentclass[a4paper]{article}

\usepackage[english]{babel}
\usepackage[utf8]{inputenc}
\usepackage{amsmath}
\usepackage{graphicx}
\usepackage[colorinlistoftodos]{todonotes}
\usepackage{listings}
\addtolength{\textwidth}{1in}
\addtolength{\oddsidemargin}{-.5in}

\title{System Validation 2016 \\ Homework Part 2 - Model Checking}

\author{Koen Mulder (s1757679) and Ruben van den Berg (s1354914)}

\date{\today}

\begin{document}
	\maketitle
	
	\begin{abstract}
		This homework assignment is done for the course System Validation. 
	\end{abstract}
	
	\section{Lock Requirements}

	
	\section{SMV and Temporal Logics}
	
	
	\subsection{Properties}
	the properties we have written are visible in the \textit{\textbf{properties.smv}} file that comes with this report.
	During the creation of these properties we made multiple assumptions, these we will mention in this section.
	
	One of the assumptions we made was about the track parts \texttt{T4A} and \texttt{T4B}. According to the specifications it is allowed to move onto this piece of track as there is no mention otherwise, also according to the image there seems to be a relative big piece of track compared to the other switch tracks \texttt{T2A} and \texttt{T2B}. Based on this we made the assumption that the signals \texttt{S4A} and \texttt{S4B} can be green if they are locked straight. In this case a requirement similar to:
	\begin{lstlisting}
LTLSPEC G((T4A_occupied & P2A_locked_straight) -> S4A_red)
LTLSPEC G(((!P1B_locked_curved | T2B_occupied | T1B_occupied | T4A_occupied) & P2A_locked_curved) -> S4A_red)
LTLSPEC G((!P2A_locked_straight & !P2A_locked_curved) -> S4A_red)
	\end{lstlisting}
	Would be replaced with:
	\begin{lstlisting}
LTLSPEC G(((!P1B_locked_curved | T2B_occupied | T1B_occupied | T4A_occupied) & P2A_locked_curved) -> S4A_red)
LTLSPEC G(!P2A_locked_curved -> S4A_red)
	\end{lstlisting}
	
	A situation where we partly deviated from the specifications is in the case of \textit{\textbf{The points always follow the given commands}}. This partly has to do with our interpretation of the specification and with the functions of the system. In our interpretation we interpreted this requirement as when a goal is given it has to follow the order before having a new order. However this will not work in a situation with multiple trains. In some situations a certain order has priority over an other one. This is the case with 2 trains at \texttt{T1A} and \texttt{T3B}. In the example first a train would arrive at \texttt{T3B}, setting the goal of 2 points to curved. However then a train arrives at \texttt{T1A}, the goals of the points are set to straight while their never entered a locked curved position.
	Because of this situation we added this as an exception. The a goal curved can be replaced by a goal straight, as long as it will still eventually be locked curved. However depending on the interpretation of the informal requirements this still can be correct behavior.
	
	\subsection{Error case counter example}
	
	
	\subsection{Delay case}
	
	
	\subsection{Full case}
	Notice: we did not use the delay case
	

	\section{Software Model Checking}


	\subsection{Asserts and assumptions}
	
	
	\subsection{The 2 case}
	
	
	\subsection{The minus plus switch case}
	
	
	
\end{document}
