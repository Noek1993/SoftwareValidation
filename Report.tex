\documentclass[a4paper]{article}

\usepackage[english]{babel}
\usepackage[utf8]{inputenc}
\usepackage{amsmath}
\usepackage{graphicx}
\usepackage[colorinlistoftodos]{todonotes}
\usepackage{listings}

\title{System Validation 2016 \\ Homework Part 1 - JML}

\author{Koen Mulder (s1757679) and Ruben van den Berg (s1354914)}

\date{\today}

\begin{document}
\maketitle

\begin{abstract}
This homework assignment is done for the course System Validation. Here has been practiced with JML Annotations, Runtime Assertion Checking(from now on called RAC), Static Checking and Test Generation. In the first part JML annotations were made for informal array requirements. In the second part a game of Sokoban was used where informal requirements were made formal and were checked by JML Annotations, RAC, Static Checking and Test Generation. In the end the program behaved accordingly to the informal requirements. 
\end{abstract}

\section{Array Requirements}
The first exercise relates to the array specification exercise as given in the first exercise session of the course. Below all informal requirements are given and a JML specification to solve this.
Consider an array $B$ with $n$ elements: $B[0], . . . , B[n-1]$.
All elements in $B$ are integers. Let $j$, $k$ be two indices such that $0 \leq j < k < n$.
With $B[j], . . . , B[k]$ we denote the segment of $B$ starting from index $j$ and ending with index $k$.
\begin{lstlisting}
	/*@ spec_public */ private int[] B;	// The array
	/*@ spec_public */ private int n;	// Lenght of the array
	/*@ spec_public */ private int j;	// Index start
	/*@ spec_public */ private int k;	// Index end
	
	//@ public invariant j >= 0 && j < k && k < n;
\end{lstlisting}
\section{JML Annotations and Runtime Assertion Checking}


\section{Static Checking}


\section{Test Generation}



\end{document}
