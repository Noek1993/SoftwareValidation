\documentclass[a4paper]{article}

\usepackage[english]{babel}
\usepackage[utf8]{inputenc}
\usepackage{amsmath}
\usepackage{graphicx}
\usepackage[colorinlistoftodos]{todonotes}
\usepackage{listings}
\addtolength{\textwidth}{1in}
\addtolength{\oddsidemargin}{-.5in}

\title{System Validation 2016 \\ Homework Part 1 - JML}

\author{Koen Mulder (s1757679) and Ruben van den Berg (s1354914)}

\date{\today}

\begin{document}
\maketitle

\begin{abstract}
This homework assignment is done for the course System Validation. Here has been practiced with JML Annotations, Runtime Assertion Checking(from now on called RAC), Static Checking and Test Generation. In the first part JML annotations were made for informal array requirements. In the second part a game of Sokoban was used where informal requirements were made formal and were checked by JML Annotations, RAC, Static Checking and Test Generation. In the end the program behaved accordingly to the informal requirements. 
\end{abstract}

\section{Array Requirements}
The first exercise relates to the array specification exercise as given in the first exercise session of the course. Below all informal requirements are given and a JML specification to solve this.
\begin{itemize}
\item Consider an array $B$ with $n$ elements: $B[0], . . . , B[n-1]$.
All elements in $B$ are integers. Let $j$, $k$ be two indices such that $0 \leq j < k < n$.
With $B[j], . . . , B[k]$ we denote the segment of $B$ starting from index $j$ and ending with index $k$. Which is in JML specifications
\begin{lstlisting}
/*@ spec_public */ private int[]B;	// The array
/*@ spec_public */ private int n;	// Lenght of the array
/*@ spec_public */ private int j;	// Index start
/*@ spec_public */ private int k;	// Index end
	
//@ public invariant j >= 0 && j < k && k < n;
\end{lstlisting}

	\item Any value that occurs in the segment $B[j], . . . ,B[k]$ also occurs outside that segment.
	\begin{lstlisting}
//@ public invariant (\forall int i1; i1 >= j && i1 <= k;
(\exists int i2; (i2 >= 0 && i2 < j) || (i2 > k && i2 < n); B[i1] == B[i2]));
	\end{lstlisting}
	This is checked by going over this segment and see if there is a match for every element outside of $B[j], . . . , B[k]$
	
	\item The array $B$ does not contain duplicates when the segment $B[j], . . . , B[k]$ is not considered.
	\begin{lstlisting}
//@ public invariant (\forall int i1, i2; i1 >= 0 && i2 > i1 && i2 < n;
(B[i1] == B[i2]) ==> ((i1 >= j && i1 <= k) || (i2 >= j && i2 <= k)));
	\end{lstlisting}
	If there are two elements who are duplicates of each other it must mean one of the elements is inbetween $B[j], . . . , B[k]$.

	\item Any value occurs at most twice in B.
	\begin{lstlisting}
//@ public invariant (\forall int i1,i2,i3; i1>=0 && i2>i1 && i3>i2 && i3<n;
B[i1] != B[i2] || B[i2] != B[i3] || B[i1] != B[i3]);	
	\end{lstlisting}
	If you have three values at least two values can not be equal to eachother.
	
	\item The highest value occuring in B occurs at least twice in B.
	\begin{lstlisting}
//@ public invariant (\exists int i1, i2; i1 >= 0 && i2 > i1 && i2 < n;
B[i1] == B[i2] && (\forall int i3; i3 >= 0 && i3 < n; B[i1] >= B[i3]));	
	\end{lstlisting}
	First we check if two elements are the same. If so we check for all other elements if they are equal or higher then those elements.
	
	\item The array B contains a palindrome of length k.
	\begin{lstlisting}
//@ public invariant (\exists int i1, i2; i1 >= 0 && i2 == (i1+k-1) && i2 < n;
(\forall int i3; i3 >= 0 && i3 < k; B[i1 + i3] == B[i2 - i3]));	
	\end{lstlisting}
	Index i1 and i2 are a segement with lenght k, after this we check for the palindrome. We consider the segment i1,i2 a sub array. We use two indices to check from back to front and the other way around. For every possible itteration the elements for these indices should be the same.
\end{itemize}

\section{JML Annotations and Runtime Assertion Checking}


\section{Static Checking}


\section{Test Generation}



\end{document}
